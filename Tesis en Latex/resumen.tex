\newpage
\textbf{\LARGE Resumen}

  El contenido de este trabajo est\'{a} centrado en la demostraci\'{o}n de la superioridad que existe entre los modelos de aprendizaje profundo y los modelos de aprendizaje autom\'{a}tico dentro de la inteligencia artificial. Para ello se realizar\'{a} una fundamentaci\'{o}n previa de los conceptos b\'{a}sicos y herramientas utilizadas para realizar las experimentaciones. Se presentar\'{a}n las estructuras de los modelos para realizar las demostraciones, adem\'{a}s de seleccionar los mejores modelos para el desarrollo de la herramienta para la detecci\'{o}n de fraude de tarjeta de cr\'{e}dito.
  
\textbf{\small Palabras clave: modelos de aprendizaje profundo, modelos de aprendizaje autom\'{a}tico, detecci\'{o}n de fraude de tarjeta de cr\'{e}dito, inteligencia artificial.}\\

\textbf{\LARGE Abstract}\\\\

The content of this work is focused on the demonstration of the superiority that exists between deep learning models and machine learning models within artificial intelligence. For this, a preliminary foundation of the basic concepts and tools used to carry out the experiments is carried out. The structures of the models will be presented to carry out the demonstrations, in addition to selecting the best models for the development of tools for detecting credit card fraud.

\textbf{\small Keywords: deep learning models, machine learning models, credit card fraud detection, artificial intelligence}\\